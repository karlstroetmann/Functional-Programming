\chapter{Introduction}
In this introduction I will do two things:
\begin{enumerate}
\item First, I discuss those features of Haskell that set Haskell apart from other programming languages.
\item Second, I will present a few short example programs that give a first taste of Haskell.
\end{enumerate}

\section{Why Haskell is Different}
Before we present any details of \textsl{Haskell}, let us categorize this programming language so that we have
an idea about what to expect.  \textsl{Haskell} has the following properties:
\begin{enumerate}
\item \textsl{Haskell} is a \blue{functional programming language}.

  A functional programming language is any programming language that treats functions as
  \blue{first class citiziens}:
  \begin{enumerate}
  \item A function can be given as an argument to another function.
  \item A function can return a function as its result.
  \end{enumerate}
  Of course, Haskell is not the only functional programming language.
  \begin{enumerate}
  \item The first functional programming language is \blue{Lisp} (Tiobe index: 23), which is short for
        \textit{\underline{Lis}t \underline{P}rocessing}. 
        It was developed in 1958 by John McCarthy at the Massachusetts Institute of Technology (MIT). 
        Lisp introduced many groundbreaking concepts, such as \blue{garbage collection}, \blue{dynamic typing},
        and \blue{higher-order functions}. A dialect of Lisp is \blue{Scheme}, which is a typed version of Lisp.
  \item \href{https://en.wikipedia.org/wiki/ML_(programming_language)}{ML} (Tiobe index: 46), short for
        \textit{Meta Language}, is a general-purpose functional programming language 
        developed in the 1970s by Robin Milner and others at the University of Edinburgh. It was originally
        designed as part of the \textit{LCF theorem-proving system} to facilitate automated reasoning and formal
        verification.  ML introduced several influential features, including static type inference, pattern
        matching, and parametric polymorphism.  An important dialect of ML is
        \href{https://en.wikipedia.org/wiki/OCaml}{OCaml} (objective Caml, Caml is short for Cambridge ML).  ML
        is widely used in academia and 
        industry for compiler design, formal methods, and functional programming.
  \item \href{https://en.wikipedia.org/wiki/F_Sharp_(programming_language)}{\blue{\texttt{F\#}}} is a strongly-typed,
        functional-first programming language developed by Microsoft Research 
        and first released in 2005. It is part of the \href{https://en.wikipedia.org/wiki/.NET_Framework}{.NET}
        ecosystem and combines functional, imperative, and object-oriented programming paradigms.  \texttt{F\#}
        is influenced by OCaml and other functional languages, offering 
        features such as type inference, pattern matching, immutability, and asynchronous programming. It also
        supports interoperability with other .NET languages like
        \href{https://en.wikipedia.org/wiki/C_Sharp_(programming_language)}{\texttt{C\#}} and 
        \href{https://en.wikipedia.org/wiki/.NET_Framework}{\texttt{VB.NET}}. \texttt{F\#} is widely used for data
        analysis, scientific computing, and financial modeling due to its concise syntax and powerful
        abstractions.

        \texttt{F\#} has been designed by Don Syme.
        
  \item \href{https://en.wikipedia.org/wiki/Scala_(programming_language)}{Scala}
        a modern programming language launched in 2004 by Martin Odersky is notable for the following features:
        \begin{itemize}
        \item \textsl{Scala} is a multi-paradigm langauge as it supports both functional and object-oriented
              programming. 
        \item It runs on the Java Virtual Machine and therefore allows seamless use of Java libraries and
              tools. Therefore it enables calling Java code and being called by Java.
        \item Similar to \textsl{Haskell} it features a strong, static type system with type inference.
        \item It includes support for concurrency.
        \item It uses a concise syntax with reduced boilerplate code, although the syntax is not as concise as
              the syntax of \textsl{Haskell}.
        \end{itemize}
  \item Many modern programming languages support functional programming.  For example,
        \href{https://en.wikipedia.org/wiki/Python_(programming_language)}{Python} (Tiobe index: 1) 
        can be used as a functional programming language.  It supports higher order functions, garbage
        collection, matching, lambda expressions, list comprehensions, closures, iterators, decorators, and
        provides libraries like \texttt{functools} and \texttt{itertools} that support functional programming.
        The following books discuss functional programming in \textsl{Python}: \cite{lott:2022}, \cite{mertz:2015},
        and \cite{reid:2023}.         
  \end{enumerate}
\item \textsl{Haskell} is \blue{statically typed}.

  Every variable in Haskell has a fixed type, which can not be changed.  In this respect, \textsl{Haskell} is
  similar to the programming language \textsl{Java}.  However, in contrast to \textsl{Java}, we do not have to
  declare the type of every variable and every function because most of the time the type of a variable can be
  \blue{inferred} by the type system.  Therefore, in Haskell we usually specify only the types of non-trivial
  functions.

  The benefit of this approach is that \textsl{Haskell} programs are much more concise than the corresponding
  \textsl{Java} programs, while the type errors will still be caught by the compiler.  This is in
  contrast to programs written in a dynamically typed language like \textsl{Python}: Similar to
  \textsl{Haskell} programs, \textsl{Python} programs are quite concise, but type errors are only
  discovered at runtime.  Therefore, programming in \textsl{Python} is quite unsafe when compared to
  programming in \textsl{Haskell}.

\item \textsl{Haskell} is a \blue{pure} functional programming language.

  Once a variable is assigned a value, this value can not be changed.  For example, if we want to sort a list,
  we are not able to change the list data structure.  All we can do is to compute a new list which contains the
  same elements as the old list and which, furthermore, is sorted.

  The property of being a \blue{pure} language sets \textsl{Haskell} apart from most other programming
  languages.
  
  What is the big deal about purity?  On one hand, it forces
  the user to program in a declarative style.  Although, in general, nobody likes to be forced to do something,
  there is a huge benefit in pure programming.
  \begin{enumerate}
  \item In a pure programming language, functions will always return the same result when they are called with
    the same arguments.  This property is called \blue{referential transparency}.
    This makes reasoning about code easier, as you can replace a function call with its result without changing
    the behavior of the program.  Therefore the correctness of functions can be verified mathematically.
  \item Since pure functions do not depend on or modify external state, their behavior is entirely predictable.
    The advantage is that testing becomes straightforward because functions can be tested in isolation without
    worrying about interactions with other parts of the system. 
  \item Compilers for pure languages can make aggressive optimizations, such as caching function results
    (\blue{memoization}) or reordering computations, because they know that functions are side-effect-free. 
    The advantage is that programs can often run faster.
  \item Furthermore, \blue{concurrency} becomes much easier to manage when functions do not use global variables and do
    not change their arguments.
  \end{enumerate}
  
\item \textsl{Haskell} is a \blue{compiled} language similar to \textsl{Java} and \texttt{C}, but additionally
  offers an interpreter.  Having an interpreter is beneficial for rapid prototyping.  The property that
  \textsl{Haskell} programs can be compiled ensures that the resulting programs can be faster than,
  for example, \textsl{Python} programs.

  To make this claim concrete, consider the \textsl{Python} program shown in Figure \ref{fig:fibonacci.py} on
  page \pageref{fig:fibonacci.py} to compute the
  \href{https://en.wikipedia.org/wiki/Fibonacci_numbers}{Fibonacci numbers}.  The Fibonacci sequence
  $(F_n)_{n\in\mathbb{N}}$ is defined inductively:
  \\[0.2cm]
  \hspace*{1.3cm}
  $F_0 = 0$, $F_1 = 1$, and $F_{n+2} = F_{n+1}  + F_n$
  \\[0.2cm]
  The algorithm implemented in Figure \ref{fig:fibonacci.py} is 
  deliberately inefficient.  On my Mac Studio of 2022 the \textsl{Python} program takes about 11.40 seconds to
  compute the $40^{\mathrm{th}}$ Fibonacci number.

\begin{figure}[!ht]
\centering
\begin{minted}[ frame         = lines, 
                 framesep      = 0.3cm, 
                 firstnumber   = 1,
                 bgcolor       = bg,
                 numbers       = left,
                 numbersep     = -0.2cm,
                 xleftmargin   = 0.8cm,
                 xrightmargin  = 0.8cm,
               ]{python3}
    def fib(n):
        if n == 0:
            return 0
        elif n == 1:
            return 1
        else:
            return fib(n - 1) + fib(n - 2)
    
    if __name__ == "__main__":
        n = 40  # Calculate the 40th Fibonacci number
        print(fib(n))
\end{minted}
\vspace*{-0.3cm}
\caption{Computing the Fibonnacci numbers in Python.}
\label{fig:fibonacci.py}
\end{figure}

    In contrast, Figure \ref{fig:fibonacci.hs} on page \pageref{fig:fibonacci.hs} shows a \textsl{Haskell}
    implementation of the same inefficient algorithm.  If we compile this program with
    \\[0.2cm]
    \hspace*{1.3cm}
    \texttt{ghc -O2 fibonacci.hs}
    \\[0.2cm]
    then the program runs in 0.62 seconds, which is more than 18 times faster.

    Admittedly, the \texttt{C} implementation runs in 0.33 seconds and is about twice as fast as
    \textsl{Haskell}.  A \textsl{Java} implementation is as fast as the \texttt{C} implementation.
\begin{figure}[!ht]
\centering
\begin{minted}[ frame         = lines, 
                 framesep      = 0.3cm, 
                 firstnumber   = 1,
                 bgcolor       = bg,
                 numbers       = left,
                 numbersep     = -0.2cm,
                 xleftmargin   = 0.8cm,
                 xrightmargin  = 0.8cm,
               ]{haskell}
    fib :: Int -> Int
    fib 0 = 0
    fib 1 = 1
    fib n = fib (n - 1) + fib (n - 2)
    
    main :: IO ()
    main = do
        let n = 40  -- Calculate the 40th Fibonacci number
        print (fib n)
\end{minted}
\vspace*{-0.3cm}
\caption{Computing the Fibonnacci numbers in Haskell}
\label{fig:fibonacci.hs}
\end{figure}

  
\item \textsl{Haskell} is a \blue{lazy} language.  In contrast, most other programming languages like
  \texttt{C}, \textsl{Python}, and \textsl{Java} use an \blue{eager} evaluation strategy, which works as follows:
  If an expression  of the form
  \\[0.2cm]
  \hspace*{1.3cm}
  $f(a_1, \cdots, a_n)$
  \\[0.2cm]
  has to be evaluated, first the subexpressions $a_1$, $\cdots$ $a_n$ are evaluated.  Let us assume that $a_i$
  is evaluated to to value $x_i$.  Then, $f(x_1, \cdots, x_n)$ is computed.  This might be very inefficient.
  Consider the example shown in Figure \ref{fig:eager.c} on page \pageref{fig:eager.c}.

\begin{figure}[!ht]
  \centering
\begin{minted}[ frame         = lines, 
                 framesep      = 0.3cm, 
                 firstnumber   = 1,
                 bgcolor       = bg,
                 numbers       = left,
                 numbersep     = -0.2cm,
                 xleftmargin   = 0.8cm,
                 xrightmargin  = 0.8cm,
               ]{C}
  int f(int x, int y) {
      if (x == 2) {
          return 42;
      }
      return 2 * y;
  }
\end{minted}
\vspace*{-0.3cm}
\caption{A \texttt{C} program.}
\label{fig:eager.c}
\end{figure}

Let us assume that the expression
\\[0.2cm]
\hspace*{1.3cm}
\texttt{f(h(0), g(1))}
\\[0.2cm]
needs to be evaluated and that the computation of \texttt{g(1)} is very expensive.  In a
\texttt{C}-program, the expressions and \texttt{h(0)} and \texttt{g(1)} will be both evaluated.  If it turns out that
\texttt{h(0)} is \texttt{2}, then the evaluation of \texttt{g(1)} is not really necessary.  Nevertheless, in \texttt{C} this
evaluation takes place because \texttt{C} has an \blue{eager} evaluation strategy.  In contrast, an equivalent
\textsl{Haskell} program would not evaluate the expression \texttt{h(0)} and hence would be much more efficient.
\item \textsl{Haskell} has influenced the design of many modern programming languages. To take one example,
      the programming language \href{https://en.wikipedia.org/wiki/Rust_(programming_language)}{Rust} has been
      heavyly influenced by Haskell as acknowledged by the Rust developers on this
      \href{https://doc.rust-lang.org/reference/influences.html}{site}.
      \begin{enumerate}
      \item The \blue{type classes} and \blue{type families} are implemented in Rust as
            \blue{traits}.
      \item \textsl{Haskell} has inherited algebraic data types, pattern matching, and type inference from
            the functional programming language ML discussed above.  These features have also found their way
            into Rust
      \end{enumerate}
\item \textsl{Haskell} is difficult to learn.

  You might ask yourself why \textsl{Haskell} hasn't been adopted more widely.  After all, it has all these
  cool features mentioned above.  The reason is that learning \textsl{Haskell} is a lot more difficult then
  learning a language like \textsl{Python} or \textsl{Java}.  
  \begin{enumerate}[(a)]
  \item First, \textsl{Haskell} differs a lot from those languages that most people know.
  \item In order to be very concise, the syntax of \textsl{Haskell} is quite different from the syntax of
        established programming languages.
  \item \textsl{Haskell} requires the programmer to think on a very high level of abstraction.
        Many students find this difficult.
  \item Lastly, and most importantly, \textsl{Haskell} supports the use of a number of concepts
        like, e.g.~\href{https://en.wikipedia.org/wiki/Functor}{functors},
        \href{https://en.wikipedia.org/wiki/Applicative_functor}{applicatives}, and
        \href{https://en.wikipedia.org/wiki/Monad_(category_theory)}{monads} from 
        \href{https://en.wikipedia.org/wiki/Category_theory}{category theory}. 
        It takes both time and mathematical maturity to grasp these concepts.

        If you really want to understand the depth of \textsl{Haskell}, you have to dive into those topics.
        That said, while it is beneficial to understand functors, applicatives, and monads, you do not have to
        understand category theory.
  \item Fortunately, it is possible to become productive in \textsl{Haskell} without understanding 
        functors, applicatives, and monads.  Therefore, this lecture will focuss on those parts of
        \textsl{Haskell} that are more easily accessible. 
  \end{enumerate}
\end{enumerate}
Haskell was first released in 1990. It was developed by a committee of researchers.  The language has since
evolved through multiple versions, with Haskell 98 and Haskell 2010 serving as major milestones.

\section{A First Taste of Haskell}
To get a taste of Haskell we will start by computing the set of all prime numbers.
A \href{https://en.wikipedia.org/wiki/Prime_number}{prime number} is a natural number $p$ that is different from $1$
and that can not be written as a product of two natural numbers $a$ and $b$ that are both different from $1$.  If
we denote the set of all prime numbers with the symbol $\mathbb{P}$, we therefore have:
\\[0.2cm]
\hspace*{1.3cm}
$\ds \mathbb{P} = \bigl\{ p \in \mathbb{N} \mid n \not= 1 \wedge \forall a, b \in \mathbb{N}:(a \cdot b = p
\implies a = 1 \vee b = 1) \bigr\}$
\\[0.2cm]
An efficient method to compute the prime numbers is the
\href{https://en.wikipedia.org/wiki/Sieve_of_Eratosthenes}{sieve of Eratosthenes}.
This is an algorithm used to find all primes up to a given number. The method works by iteratively marking the
multiples of each prime number starting from 2. The numbers which remain unmarked at the end of the process are
the prime numbers.  For example, consider the list of integers from 2 to 30:
\\[0.2cm]
\hspace*{0.3cm}
$2, 3, 4, 5, 6, 7, 8, 9, 10, 11, 12, 13, 14, 15, 16, 17, 18, 19, 20, 21, 22, 23, 24, 25, 26, 27, 28, 29, 30$.
\\[0.2cm]
We will now compute the set of all primes less or equal than 30 using the Sieve of Eratosthenes.
\begin{enumerate}
\item The first number in this list is $2$ and hence $2$ is prime:
      \\[0.2cm]
      \hspace*{1.3cm}
      $\mathbb{P} = \{ 2, \cdots \}$.
\item We remove all multiples of 2 (i.e., 2, 4, 6, 8, \(\ldots\), 30).  This leaves us with the list
      \\[0.2cm]
      \hspace*{1.3cm}
      $3, 5, 7, 9, 11, 13, 15, 17, 19, 21, 23, 25, 27, 29$.
      \\[0.2cm]
      The first remaining number $3$ is prime.  Hence we have
      \\[0.2cm]
      \hspace*{1.3cm}
      $\mathbb{P} = \{ 2, 3, \cdots \}$
\item We remove all multiples of 3 (i.e., 3, 6, 9, 12, \(\ldots\), 27).  Note that some numbers (like 6 or 12,
      for instance) may already have been removed.  This leaves us with the list
      \\[0.2cm]
      \hspace*{1.3cm}
      $5, 7, 11, 13, 17, 19, 23, 25, 29$.
      \\[0.2cm]
      The first remaining number $5$ is prime. Hence we have
      \\[0.2cm]
      \hspace*{1.3cm}
      $\mathbb{P} = \{ 2, 3, 5, \cdots \}$.
\item We remove all multiples of 5 (i.e., 5, 10, 15, 20, \(\ldots\)). This leaves us with the list
      \\[0.2cm]
      \hspace*{1.3cm}
      $7, 11, 13, 17, 19, 23, 29$.
      \\[0.2cm]
      The first remaining number $7$ is prime. Hence we have
      \\[0.2cm]
      \hspace*{1.3cm}
      $\mathbb{P} = \{ 2, 3, 5, 7, \cdots \}$.
\item Since the square of $7$ is $49$, which is greater that 30, there is no need to remove the multiples of
      $7$,  since all multiples $a \cdot 7$ for $a < 7$ have already been removed and all multiples $a \cdot 7$
      for $a \geq 7$ are greater than 30. Hence the remaining numbers are all prime and we have found that the
      prime numbers less or equal than 30 are: 
      \\[0.2cm]
      \hspace*{1.3cm}
      $\{ 2, 3, 5, 7, 11, 13, 17, 19, 23, 29\}$. 
\end{enumerate}

\begin{figure}[!ht]
\centering
\begin{minted}[ frame         = lines, 
                 framesep      = 0.3cm, 
                 firstnumber   = 1,
                 bgcolor       = bg,
                 numbers       = left,
                 numbersep     = -0.2cm,
                 xleftmargin   = 0.8cm,
                 xrightmargin  = 0.8cm,
               ]{python3}
    def sieve_of_eratosthenes(n):
        """Return a list of prime numbers up to n (inclusive)."""
        if n < 2:
            return []    
        is_prime = [True] * (n + 1)
        is_prime[0] = is_prime[1] = False  # 0 and 1 are not primes
        p = 2
        while p * p <= n:
            if is_prime[p]:
                for i in range(p * p, n + 1, p):
                    is_prime[i] = False
            p += 1
        return [i for i, prime in enumerate(is_prime) if prime]
\end{minted}
\vspace*{-0.3cm}
\caption{A \textsl{Python program to compute the prime numbers up to $n$.}}
\label{fig:primes.py}
\end{figure}
Figure \ref{fig:primes.py} on page \pageref{fig:primes.py} shows a Python script that implements the Sieve of
Eratosthenes to compute all prime numbers up to a limit \texttt{n}.  This script first initializes the list
\texttt{is\_prime} to track prime status. It then iteratively marks the multiples of each prime number, finally
collecting and printing all numbers that remain marked as prime.  This script implements one optimization: If
\texttt{p} is a prime, then only the multiples of \texttt{p} that have the form $a \cdot \mathtt{p}$ with
$a \geq \mathtt{p}$ have to be removed from the list, since a product of the form $a \cdot \mathtt{p}$ with
$a < \mathtt{p}$ has already been removed when removing multiples  
of $a$ or, if $a$ is not prime, of multiples of whatever prime is contained in $a$.



\begin{figure}[!ht]
\centering
\begin{minted}[ frame         = lines, 
                 framesep      = 0.3cm, 
                 firstnumber   = 1,
                 bgcolor       = bg,
                 numbers       = left,
                 numbersep     = -0.2cm,
                 xleftmargin   = 0.8cm,
                 xrightmargin  = 0.8cm,
               ]{haskell}
    primes :: [Integer]
    primes = sieve [2..]
    
    sieve :: [Integer] -> [Integer]
    sieve (p:ns) = p : sieve [n | n <- ns, mod n p /= 0]
\end{minted}
\vspace*{-0.3cm}
\caption{Computing the prime numbers in \textsl{Haskell}.}
\label{fig:primes.hs}
\end{figure}

Figure \ref{fig:primes.hs} on page \pageref{fig:primes.hs} shows a \textsl{Haskell} program to compute \textbf{all}
primes.  Yes, you have read that correct.  It doesn't compute the primes up to a given number, but rather it
computes \underline{all} primes.

The first thing to note is that line 1 and line 4 are type annotations.  They have only
been added to aid us in understanding the program.  If we would drop these lines, the program would still
work.  Hence, we have an efficient 2-line program to compute the prime numers.  Lets discuss this program line
by line. 
\begin{enumerate}[(a)]
\item Line 1 states that the function \texttt{primes} returns a list of \texttt{Integer}s.
  \texttt{Integer} is the type of all arbitrary precision integers.  The fact that we have enclosed
  the type name \texttt{Integer} in the square brackets ``\texttt{[}'' and ``\texttt{]}'' denotes that the
  result has the type \textsl{list} of \texttt{Integer}.
\item In line 2, the expression ``\texttt{[2..]}''  denotes the list of all integers starting from $2$.  Since
  \textsl{Haskell} is lazy, it is able to support infinite data structures.  The trick is that these lists are
  only evaluated as much as they are needed.  As long as we do not inspect the complete list, everything works
  fine. 
\item Line 2 calls the function \texttt{sieve} with the argument \texttt{[2..]}.  \textsl{Haskell} uses prefix
  notation for calling a function.  If $f$ is a function an $a_1$, $a_2$, and $a_3$ are arguments of this
  function, then the invocation of $f$ with these arguments is written as
  \\[0.2cm]
  \hspace*{1.3cm}
  $f$ $a_1$ $a_2$ $a_3$
  \\[0.2cm]
  \textbf{Note} that the expression
  \\[0.2cm]
  \hspace*{1.3cm}
  $f(a_1, a_2, a_3)$
  \\[0.2cm]
  denotes something different:  This expression would apply the function $f$ to a single argument, which is the
  triple $(a_1, a_2, a_3)$.
\item Line 4 declares the type of the function \texttt{sieve}.  This function takes one argument, which is a
  list of \texttt{Integer}s and returns a list of \texttt{Integer}s.
\item Line 5 defines the function \texttt{sieve} that takes a list of numbers $l$ that has the following
  properties:
  \begin{itemize}
  \item The list $l$ is a sorted ascendingly.
  \item If $p$ is the first element of the list $l$, then $p$ is a prime number.
  \item The list $l$ does not contain multiples of any number $q$ that is less than $p$:
    \\[0.2cm]
    \hspace*{1.3cm}
    $\forall q \in \mathbb{N}: \bigl(q < p \rightarrow \forall n \in \mathbb{N}: n \cdot q \not\in l\bigr)$.
  \end{itemize}
  Given a list $l$ with these properties, the expression
  \\[0.2cm]
  \hspace*{1.3cm}
  \texttt{sieve l}
  \\[0.2cm]
  returns a list of all prime number that are greater or equal than $p$, where $p$ is the first element of $l$: 
  \\[0.2cm]
  \hspace*{1.3cm}
  $\mathtt{sieve}\;l = [\; q \in l \mid q \in \mathbb{P}\;]$.
  \\[0.2cm]
  Hence, when \texttt{sieve} is called with the list of all natural number greater or equal than $2$
  it will return the list of all prime numbers, since $2$ is the smallest  prime numer.

  There is a lot going on in the definition of the function \texttt{sieve}.  We will discuss this definition
  in minute detail.
  \begin{enumerate}[1.]
  \item The function \texttt{sieve} is defined via \blue{matching}.  We will discuss matching in more detail in
    the next chapter.  For now we just mention that the expression
    \\[0.2cm]
    \hspace*{1.3cm}
    \texttt{(p:ns)}
    \\[0.2cm]
    matches a list with first element \texttt{p}.  The remaining elements are collected in the list
    \texttt{ns}.  For example, if \texttt{l = [2..]}, i.e.~if $l$ is the list of all natural numbers greater or
    equal than $2$, then the variable \texttt{p} is bound to the number $2$, while the variable \texttt{ns} is
    bound to the list of all natural numbers greater or equal than $3$, i.e.~\texttt{ns = [3..]}.
  \item The right hand side of the function definition, i.e.~the part after the symbol ``\texttt{=}'' defines the
    value that is computed by the function \texttt{sieve}.  This value is computed by calling the function
    ``\texttt{:}'', which is also known as the \blue{cons} function because it \underline{cons}tructs a list.
    The operator ``\texttt{:}'' takes two arguments.  The first argument is a value \texttt{u} of some type $a$
    and the second argument \texttt{us} is list of elements of the same type $a$.  An expression of the form
    \\[0.2cm]
    \hspace*{1.3cm}
    \texttt{u :\;us}
    \\[0.2cm]
    then returns a list where \texttt{u} is the first elements and \texttt{us} are the remaining elements.
    For example, we have
    \\[0.2cm]
    \hspace*{1.3cm}
    \texttt{1 :\;[2, 3, 4] = [1, 2, 3, 4]}.
  \item The recursive invocation of the function sieve takes a \blue{list comprehension} as ist first argument.
    the expression
    \\[0.2cm]
    \hspace*{1.3cm}
    \texttt{[n | n <- ns, mod n p /= 0]}
    \\[0.2cm]
    computes the list of all those number \texttt{n} from the list \texttt{ns} that have a non-zero remainder
    when divided by the prime number \texttt{p}, i.e.~this list contains all those number from the list
    \texttt{ns} that are not multiples of \texttt{p}.  There are two things to note here concerning the syntax
    of \textsl{Haskell}:
    \begin{itemize}
    \item Functions are written with prefix notation.  For example, we first write the function name
      \texttt{mode} followed by the arguments \texttt{m} and \texttt{p}.  In \textsl{Python} the expression
      \\[0.2cm]
      \hspace*{1.3cm}
      \texttt{mod m p}
      \\[0.2cm]
      would have been written as \ \texttt{m \% p}.
    \item The operator ``\texttt{/=}'' expresses inequality, i.e.~the \textsl{Haskell} expression
      \\[0.2cm]
      \hspace*{1.3cm}
      \texttt{a /= b} \qquad would be written as \qquad \texttt{a != b} 
      \\[0.2cm]
      in the programming language \textsl{Python}.
    \end{itemize}
    Putting everything together, the \textsl{Haskell} expression 
    \\[0.2cm]
    \hspace*{1.3cm}
    \texttt{[n | n <- ns, mod n p /= 0]}
    \\[0.2cm]
    is therefore equivalent to the \textsl{Python} expression
    \\[0.2cm]
    \hspace*{1.3cm}
    \texttt{[n for n in ns if n \% p != 0]}.
  \end{enumerate}
\end{enumerate}

%%% Local Variables:
%%% mode: latex
%%% TeX-master: "haskell"
%%% End:
